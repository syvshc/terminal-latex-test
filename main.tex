% latexmk -xelatex -shell-escape main
\documentclass{ctexart}
\usepackage{minted}
\setminted{
    autogobble  =   true,
    frame       =   leftline,
    fontsize    =   \small
}
% \setmintedinline{ fontsize = \normalsize }

\usepackage[inline]{enumitem}
\setlist[enumerate, 1]{label = \arabic*. }
\setlist[enumerate, 2]{label = (\arabic*)}
\usepackage{geometry}
\geometry{margin = 2.5cm}

\usepackage[os=win]{menukeys}
\usepackage{hologo}
\newcommand{\mt}[1]{\mintinline{text}{#1}}
\newcommand{\mb}[1]{\mintinline{bash}{#1}}

\title{命令行测试}
\author{Syvshc}

\begin{document}
    \maketitle
    \textbf{在下列问题中, 均将主文档名称命名为 \mt{main.tex}, 参考文献文档命名为 \mt{xampl.bib}\footnote{这是 \TeX\,Live 自带的一个参考文献库, 详见 \mb{texdoc mwe}}}
    \section{问题}
    \begin{enumerate}
        \item 使用 \mt{article} 文档类完成一篇文档, 命令行应该 (依次) 执行什么命令来进行编译?
        \begin{enumerate}
            \item 无目录, 无参考文献, 无交叉引用;
            \item 有目录, 无参考文献, 有交叉引用;
            \item 有目录, 有参考文献, 有交叉引用.
        \end{enumerate}
        \item 编译时如果报了如下错误, 应该如何退出报错界面, 或者获取可能的帮助
        \begin{minted}[]{text}
            ! Undefined control sequence.
            l.4     $ \Sum
                        _{0}^{1} $
            ?
        \end{minted}
        \item 使用 \mb{pdflatex main} 编译如下代码时, 会报出什么错误, 应如何解决
        \begin{minted}[]{latex}
            \documentclass{article}
            \usepackage{fontspec}
            \begin{document}
                test
            \end{document}
        \end{minted}
        \item 如何查看 \mb{xelatex} 的可用编译选项\footnote{在运行 \mt{xelatex -interaction=nonstopmode main} 时, \mt{-interaction=nonstopmode} 为 \mt{xelatex} 的编译选项, 它会在编译的过程中提供一些其他的功能}, 或者说如何查看 \mb{xelatex} 的帮助?
        \item 使用 \mb{pdflatex main} 编译如下代码时, 会报出什么错误, 应如何解决
        \begin{VerbatimOut}[gobble=1]{minted.doc.out}
            \documentclass{article}
            \usepackage{minted}
            \begin{document}
                \begin{minted}{python}
                    x = [for x in range(10) if x % 2 == 1]
                    print(x)
                \end{minted}
            \end{document}
        \end{VerbatimOut}
        \inputminted[]{latex}{minted.doc.out}
        \item 我想让我的 \mt{pdf} 文档与 \mt{tex} 源文件有正反向搜索的功能, 应该在编译时添加什么选项?
        \item 我在编译 \mt{main.tex} 的时候, 想让输出文件名为 \mt{zwxm.pdf} 应该在编译时添加什么选项?
        \item 如果编译时报了如下的错误, 应该如何解决, 或者说应该首先排查什么问题
        \begin{minted}[]{text}
            ! I can't write on file `main.pdf'.
            (Press Enter to retry, or Control-Z to exit; default file extension is `.pdf')
            Please type another file name for output:
        \end{minted}
        \item 使用 \mt{pdflatex + bibtex + pdflatex*2} 编译如下文档, 会报什么错\footnote{有时报错并不会以 \mt{error} 的形式出现, 如交叉引用, 参考文献处的错误可能表现为 \mt{warning}, 此时该 \mt{warning} 会影响到编译结果, 需要处理}, 应该如何解决
        \begin{minted}[]{latex}
            \documentclass{article}
            \usepackage{biblatex}
            \addbibresource{xampl.bib}
            \begin{document}
                this is a cite\cite{article-full}
                \printbibliography
            \end{document}
        \end{minted}
        \item 使用 \mt{latexmk} 可以自动地进行参考文献, 交叉引用等处理, 应该如何查看 \mt{latexmk} 相关的编译选项?
        \item 如果想使用 \mt{latexmk} 运行 \mt{pdflatex} 与 \mt{xelatex}, 应该分别添加什么编译选项?
        \item 如果想使用 \mt{latexmk} 的 ``自动检测更改并自动编译'' 的功能, 应该添加什么编译选项?
        \item 如果想使用 \mt{latexmk} 清除辅助文件, 应该运行什么命令? 同时如果想清除除源文件以外的所有输出文件, 应该运行什么命令?
    \end{enumerate}

    \section{非标准答案}
    \begin{enumerate}
        \item \begin{enumerate*}
            \item \mt{pdflatex main} 
            \item \mt{pdflatex pdflatex}
            \item \mt{pdflatex bibtex pdflatex*2}
        \end{enumerate*}
        \item 输入 \keys{? + \enter} 可以获得可行的操作, 特别地, 输入 \keys{x + \enter} 可以退出, 输入 \keys{h + \enter} 可以或许可能的帮助.
        \item 会报以下的错误 
        \begin{minted}[]{text}
            ! Fatal Package fontspec Error: The fontspec package requires either XeTeX or
            (fontspec)                      LuaTeX.
            (fontspec)
            (fontspec)                      You must change your typesetting engine to,
            (fontspec)                      e.g., "xelatex" or "lualatex"instead of
            (fontspec)                      "latex" or "pdflatex".

            Type <return> to continue.
            ...

            l.45 \msg_fatal:nn {fontspec} {cannot-use-pdftex}

            ?
        \end{minted}
        应该改用 \mt{xelatex} 或 \mt{lulatex} 进行编译.
        \item 命令行运行 \mt{xelatex --help}
        \item 会报以下错误
        \begin{minted}[]{latex}
            ! Package minted Error: You must invoke LaTeX with the -shell-escape flag.

            See the minted package documentation for explanation.
            Type  H <return>  for immediate help.
            ...

            l.3 \begin
                    {document}
            ?
        \end{minted}
        应该在编译的时候添加 \mt{-shell-escape}, 即使用 \mt{pdflatex -shell-escape main} 编译\footnote{该宏包需要调用外部程序 \mt{pygments}, 具体可以查阅 \mt{texdoc minted}}.
        \item 添加 \mt{-synctex=1} 选项.
        \item 添加 \mt{-jobname=zwxm} 选项.
        \item 错误为 \mt{main.pdf} 不能被写入, 应该先检查 \mt{main.pdf} 有没有被如 Adobe Acrobat DC, Foxit Reader 等 PDF 阅读器打开. 
        \item 会返回如下警告
        \begin{minted}[]{text}
            LaTeX Warning: There were undefined references.

            Package biblatex Warning: Please (re)run Biber on the file:
            (biblatex)                main
            (biblatex)                and rerun LaTeX afterwards.
        \end{minted}
        这是除了 \hologo{BibTeX} 以外的另一种参考文献处理方式: \textsc{Bib}\hologo{LaTeX}, 需要将 \mt{bibtex} 替换为 \mt{biber} 进行编译
        \item 可以通过命令行运行 \mt{latexmk -help} 进行查看, 或者命令行运行 \mt{texdoc latexmk} 查看其文档.
        \item \mt{-pvc}
        \item \begin{enumerate*}
            \item \mt{-c}
            \item \mt{-C}
        \end{enumerate*}
    \end{enumerate}
\end{document}